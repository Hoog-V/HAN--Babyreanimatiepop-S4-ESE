\chapter{State-of-Art}
\label{chapter:conc}
Where did the development of the embedded systems landed? What is realised, and finished etc. 
The main question that this chapter answers is: where did the embedded systems engineering development land? What is the state of the delivered work. What is realised and could be assumed to be finished.
\section{Project Results}
This section lists the modules or parts that are realised. \\
In general this is the contribution that this project had:\\
\begin{itemize}
    \item Modular firmware with proper development organizational version control.
    \item Version controlled development environment with code standards.
    \item Crucial data acquisition software modules tested.
\end{itemize}
Note: The "(RESULT)" indicates whether the result was not a must have in the requirement list.
\begin{table}[hb!]
\begin{tabular}{ |c|l|c|c| } 
 \hline
 Requirement ID & Requirement & Achieved \\ 
 \hline
 \hline
 F1    & Every sensor in the system has to provide an validated value  & (YES) \\ 
       & (within specifications listed in the datasheet of the sensor) to the sensorhub.  &  \\
       &                                                                                  &  \\
\hline
 F1.1  & Validation of the sensors has to be done with a calibrated                       & (NO) \\
       & reference or testing device.                                                     &   \\
\hline
 F2    & A Test suite has to be designed to validate and collect sensor results.          & (YES) \\
 \hline
 F2.1  & The Test suite has to have a plotting feature                                    & (NO) \\
 \hline 
 F2.2  & The Test suite has to support exporting harvested sensor data to an excel or csv file & YES \\
 \hline
 F3    & Data from the sensors has to be stored & YES\\
       & to be used later for determining whether CPR was successful.  & \\
\hline
 F4    & The software design has to support the currently chosen and constructed & YES  \\
       & sensor modules used in the Manikin & \\
 \hline
 F4.1  & The software design has to support the time of flight sensor & YES \\
       & hardware chosen and constructed by previous groups. & \\
\hline
 F4.2  & The software design has to support the pressure sensor & YES \\
       & hardware chosen and constructed by previous groups. & \\
\hline
 F4.3  & The software design has to support the differential pressure sensor & YES \\
       & hardware chosen and constructed by previous groups. & \\
\hline
 F4.4  & The software design has to support the finger position sensor & YES \\
       & hardware chosen and constructed by previous groups. & \\
\hline
 F5    & The data bus between the sensorhub and the mainbord has to use & (NO)\\
       & a testable and extendable data protocol. &   \\
 \hline
 F5.1  & The data protocol has to be implemented using the i2c bus protocol. & (NO) \\
 \hline
 F6    & The software design has to implement a post (Power-On System Test) procedure. & (NO) \\ 
 \hline
\end{tabular}
 \caption{Project requirements results}
 \label{tab:functional_requirements}
\end{table}
This project has achieved the must-haves and some extra goals of the requirements. 
% \section{System Specification}
% Technical parameters specifications of the hub. \\
% Draft of relevant parameters: sampling time (max), buffer size/max samples, memory(ram) usages current, performance?, other limitation factors/absolutes, electrical overview (why not), interfaces overview with corresponding purpose, USB service baud, pcb dimensions, etc
