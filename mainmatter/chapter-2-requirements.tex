\chapter{Functional Design}
\label{chapter:intro} 
The functional design provides a detailed description of how the system will operate and how it adds value towards the current total system solution.
\newline
This chapter describes what the system's requirements and specifications. Future more system features, and finally also the system interactions are documented.

\section{Functional Requirements}
\begin{tabular}{ |c|l|c|c| } 
 \hline
 Requirement ID & Requirement & MoSCoW \\ 
 \hline
 \hline
 F1    & Every sensor in the system has to provide an validated value  & CH \\ 
       & (within specifications listed in the datasheet of the sensor) to the sensorhub.  &  \\
       &                                                                                  &  \\
\hline
 F1.1  & Validation of the sensors has to be done with a calibrated                       & CH \\
       & reference or testing device.                                                     &   \\
\hline
 F2    & A Test suite has to be designed to validate and collect sensor results.          & CH \\
 \hline
 F2.1  & The Test suite has to have a plotting feature                                    & SH \\
 \hline 
 F2.2  & The Test suite has to support exporting harvested sensor data to an excel or csv file & SH \\
 \hline
 F3    & Data from the sensors has to be stored & MH\\
       & to be used later for determining whether CPR was successful.  & \\
\hline
 F4    & The software design has to support the currently chosen and constructed & MH  \\
       & sensor modules used in the Manikin & \\
 \hline
 F4.1  & The software design has to support the time of flight sensor & MH \\
       & hardware chosen and constructed by previous groups. & \\
\hline
 F4.2  & The software design has to support the pressure sensor & MH \\
       & hardware chosen and constructed by previous groups. & \\
\hline
 F4.3  & The software design has to support the differential pressure sensor & MH \\
       & hardware chosen and constructed by previous groups. & \\
\hline
 F4.4  & The software design has to support the finger position sensor & MH \\
       & hardware chosen and constructed by previous groups. & \\
\hline
 F5    & The data bus between the sensorhub and the mainbord has to use & SH\\
       & a testable and extendable data protocol. &   \\
 \hline
 F5.1  & The data protocol has to be implemented using the i2c bus protocol. & MH \\
 \hline
 F6    & The software design has to implement a post (Power-On System Test) procedure. & WNH \\ 
 \hline
\end{tabular}

\section{Non-Functional Requirements}
\begin{tabular}{ |c|l|c|c| } 
 \hline
 Requirement ID & Requirement & MoSCoW \\ 
 \hline
 \hline
 NF1    & The software design has to implement software engineering best practices.  & MH \\
 \hline
 NF2    & The software design has to use an consistent code style & MH \\
 \hline 
 NF3    & The software-, hardware-design and documentation are stored  & MH \\
        & in public GitHub repository's & \\
\hline
 NF4   & Software unit tests have to be constructed to test the software & MH\\
 \hline
\end{tabular}
\section{Design Constraints}
There are a few design constraints related to this project.\\ 
These constraints are either related to technical constraints or time related constraints. \\ This project is financed and backed by the school and most of the hardware is mostly handled by Johan and the IPS students, therefore there are no financial or organizational constraints.
\subsection{Technical design constraints}
There are multiple constraints in this category, due to the hardware and software decisions made by Johan and the previous project term groups. These constraints where easily pointed out and have helped with creating the functional requirements. To help analyse how much extra time a design constraint will take to tackle, a ranking system was added to the table.\\
\begin{tabular}{|l|c|}
\hline
    Constraint & Weight \\
               & (1 meaning low time effort, 10 meaning big effort) \\
    \hline\hline
     Previous groups have chosen the Arduino framework. & 5\\
     But no research was done on the subject of \\frameworks for the samd series. Which probably means that \\ custom software implementations for hardware functions \\have to be made. Due to Arduino's limited support.\\ & \\
     \hline
     Due to the varying skillset of the upcoming project term groups. & 3\\
     The code written has to be easy to understand. \\ Which might limit us in the use of tools or software frameworks.\\
     \hline
     
     
     
     
\end{tabular}

\subsection{Time related constraints}