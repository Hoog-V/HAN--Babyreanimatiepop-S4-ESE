\chapter{Introduction}
\label{chapter:intro}

\section{Background}
Each year cardiac arrests account for more than thirty percent of mortality worldwide. Since
early cardiopulmonary resuscitation (CPR) is an important factor in survival of a cardiac arrest,
it would be beneficial for a large population to have up-to-date CPR skills. This project tries to
improve quality of CPR training by proposing a system for automated real-time feedback.
Existing systems for CPR training all have their limits in providing desired feedback to a large
group of CPR learners and that is why a novel modular and open patient simulation system was
developed. \cite{jakortenmsc}\\\\
Due to the modular characteristic of this open patient simulation system, the system can be used from baby to grown up CPR training. 
For this project the goal is to implement this modular system in a baby CPR doll (also referred to as a manikin).
\section{Objective for this term project}
The previous student groups which worked on this project have made quite some progress in reading the compression, flow and finger position sensors embedded in the manikin.
\\However the code for reading the sensors are still separate programs which need to be integrated in to one big program. Besides this problem, we have to solve the varying code quality of those programs and also verify the output of these programs in a practical and scientific manner.\\\\
The objectives of this project are:\\
a). To rework every piece of written code to match code styling standards and software engineering good practices.\\
b). To integrate the reworked code in to one big program.\\
c). To test the integrated code with software and hybrid software tests (using simulators, test benches and software unit-tests).\\
d). To learn to test and integrate code 