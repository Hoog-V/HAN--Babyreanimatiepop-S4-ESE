\chapter{Recommendations}
\label{chapter:rec}

This chapter presents a comprehensive set of recommendations that outline the key areas for future development and advancement of this research project.\\
\subsubsection*{Rec 1: Optimization and Digital Signal Processing}
One significant opportunity for further investigation into the development is digital signal processing application techniques. The current system has a basic although practical digital signal processing. Research into this element of the project could be interesting for offering valuable insights and potential advancements into the feedback of the training.
\subsubsection*{Rec 2: Advancement into Security Standards}
While the current state-of-art may not require immediate security considerations, it is hightly advisable to begin contemplating and implementing security mechanisms. The development team did not implement any security techniques because this was out of project scope.  
\subsubsection*{Rec 3: Hardware Peripherals Optimization and Enhancement}
To optimize the software's performance, a recommended approach is to leverage hardware peripherals such as Direct Memory Access (DMA). By exploring the capabilities of these peripherals and effectively integrating them into the software design, it is possible to achieve significant performance improvements. This optimization strategy enables efficient data transfers, reduces processing overhead, and maximizes the utilization of available hardware resources.
\subsubsection*{Rec 4: Re-design and Development of the mainboard-hub Protocol}
The implementation of the mainboard and the hub communication protocol is not realised during this project. In appendix under section Intranet there is concept for this protocol. However it is not fully worked out. It is highly recommended to use it as a example or starting point, but not to implement directly.