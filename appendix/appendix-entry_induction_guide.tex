\chapter{Entry Induction Guide}

The goal of the Entry Induction Guide is to support next project groups to get started effectively. This guide gives the key takeaways about project specific knowledge(of concepts), skills and tools. 
\\ \\ 
Concept / Draft of things: \\
- CMakeLists (Adviceable)
- GitHub
- read/save for reference the project style guide
- Google Testing framework basics (Required, but not immediately)
- Read/Scan quickly the protocol outlines
- Etc

\section{Guide Assumptions}
The goal of the guide is to be semi-dummy proof, meaning the coverage tries to be wide covered. However due to the audience expectations the guide assumes you are familiar with the following:
\begin{itemize}
    \item Working with GitHub version control: clone, commit, push and pull. Work in branches.
    \item Have experience with VS code IDE. 
    \item Have experience and knowledge about C++.
\end{itemize}

\section{Getting Started}
The previous developers recommend that you start with following this list:
\begin{itemize}
  \item Scan quickly through Architecture Design (chapter 3), look at diagrams to get a general overview and idea of the system. 
  \item Read Firmware Description to understand how and what the firmware does.
  \item Go to the RobotPatient organization GitHub page, and get the release firmware version of Manikin Software, and see if you are able to build and upload to the target device.  
  \item Next step is to decide whether you understand the CMakeLists and Google Testing unit tests. If you have the feeling you lack some knowledge here, go to the corresponding section to find learning/guides/documentation resources.
\end{itemize}